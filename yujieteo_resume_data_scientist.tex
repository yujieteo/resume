\documentclass[letterpaper,10pt]{article}

\usepackage[empty]{fullpage}
\usepackage{titlesec}
\usepackage{enumitem}
\usepackage[hidelinks]{hyperref}
\usepackage{fancyhdr}
\usepackage{fontawesome5}
\usepackage{multicol}
\usepackage{bookmark}
\usepackage{lastpage}

% Sans-Serif
%\usepackage[sfdefault]{FiraSans}
%\usepackage[sfdefault]{roboto}
%\usepackage[sfdefault]{noto-sans}
%\usepackage[default]{sourcesanspro}

% Serif
\usepackage{CormorantGaramond}
\usepackage{charter}

% Colors
% Use with \color{Name}
% Can wrap entire heading with {\color{accent} [...] }
\usepackage{xcolor}
\definecolor{accentTitle}{HTML}{2c446f}
\definecolor{accentText}{HTML}{2c446f}
\definecolor{accentLine}{HTML}{2c446f}

% Misc. Options
\pagestyle{fancy}
\fancyhf{}
\fancyfoot{}
\renewcommand{\headrulewidth}{0pt}
\renewcommand{\footrulewidth}{0pt}
\urlstyle{same}

% Adjust Margins
\addtolength{\oddsidemargin}{-0.7in}
\addtolength{\evensidemargin}{-0.5in}
\addtolength{\textwidth}{1.19in}
\addtolength{\topmargin}{-0.7in}
\addtolength{\textheight}{1.4in}

\setlength{\multicolsep}{-3.0pt}
\setlength{\columnsep}{-1pt}
\setlength{\tabcolsep}{0pt}
\setlength{\footskip}{3.7pt}
\raggedbottom
\raggedright

% ATS Readability
\input{glyphtounicode}
\pdfgentounicode=1

%-----------------%
% Custom Commands %
%-----------------%

% Centered title along with subtitle between HR
% Usage: \documentTitle{Name}{Details}
\newcommand{\documentTitle}[2]{
  \begin{center}
    {\Huge\color{accentTitle} #1}
    \vspace{10pt}
    {\color{accentLine} \hrule}
    \vspace{2pt}
    %{\footnotesize\color{accentTitle} #2}
    \footnotesize{#2}
    \vspace{2pt}
    {\color{accentLine} \hrule}
  \end{center}
}

% Create a footer with correct padding
% Usage: \documentFooter{\thepage of X}
\newcommand{\documentFooter}[1]{
  \setlength{\footskip}{10.25pt}
  \fancyfoot[C]{\footnotes占有率 #1}
}

% Simple wrapper to set up page numbering
% Usage: \numberedPages
% WARNING: Must run pdflatex twice!
\newcommand{\numberedPages}{
  \documentFooter{\thepage/\pageref{LastPage}}
}

% Section heading with horizontal rule
% Usage: \section{Title}
\titleformat{\section}{
  \vspace{-5pt}
  \color{accentText}
  \raggedright\large\bfseries
}{}{0em}{}[\color{accentLine}\titlerule]

% Section heading with separator and content on same line
% Usage: \tinysection{Title}
\newcommand{\tinysection}[1]{
  \phantomsection
  \addcontentsline{toc}{section}{#1}
  {\large{\bfseries\color{accentText}#1} {\color{accentLine} |}}
}

% Indented line with arguments left/right aligned in document
% Usage: \heading{Left}{Right}
\newcommand{\heading}[2]{
  \hspace{10pt}#1\hfill#2\\
}

% Adds \textbf to \heading
\newcommand{\headingBf}[2]{
  \heading{\textbf{#1}}{\textbf{#2}}
}

% Adds \textit to \heading
\newcommand{\headingIt}[2]{
  \heading{\textit{#1}}{\textit{#2}}
}

% Template for itemized lists
% Usage: \begin{resume_list} [items] \end{resume_list}
\newenvironment{resume_list}{
  \vspace{-7pt}
  \begin{itemize}[itemsep=-2px, parsep=1pt, leftmargin=30pt]
}{
  \end{itemize}
  %\vspace{-2pt}
}

% Formats an item to use as an itemized title
% Usage: \itemTitle{Title}
\newcommand{\itemTitle}[1]{
  \item[] \underline{#1}\vspace{4pt}
}

% Bullets used in itemized lists
\renewcommand\labelitemi{--}

%% END_FILE: mteck.sty
%%%%%%%%%%%%%%%


%\numberedPages % NOTE: lastpage requires a second build
%\documentFooter{\thepage of 2} % Does similar without using lastpage
\begin{document}

  %---------%  
  % Heading %
  %---------%

  \documentTitle{Teo Yu Jie}{
    % Web Version
    %\raisebox{-0.05\height} \faPhone\ [redacted - web copy] ~
    %\raisebox{-0.15\height} \faEnvelope\ [redacted - web copy] ~
    %%
    \href{tel:8332 4860}{
      \raisebox{-0.05\height} 8332 4860} ~ | ~
    \href{mailto:yujie@protonmail.com}{
      \raisebox{-0.15\height} yujie@protonmail.com} ~ | ~
    \href{https://www.linkedin.com/in/yujieteo/}{
      \raisebox{-0.15\height} linkedin.com/in/yujieteo/} ~ | ~
    \href{https://github.com/yujieteo?tab=repositories}{
      \raisebox{-0.15\height} https://github.com/yujieteo?tab=repositories}
  }

  %---------%  
  % Summary %
  %---------%

  \tinysection{Summary}
  \begin{resume_list}
    \item Data Scientist specializing in machine learning, statistical analysis, and computational modeling for engineering systems.
    \item Experienced in developing ML algorithms, data pipelines, and statistical models for complex engineering and scientific applications.
    \item Skilled in deep learning, probabilistic modeling, and data-driven optimization, achieving 99.5\% performance improvements through advanced analytics.
  \end{resume_list}

    %--------%
  % Skills %
  %--------%

\section{Skills}

\begin{tabular}{@{}ll}
\textbf{Machine Learning \,\,} &  PyTorch, tinygrad, NumPy, SciPy, pandas, GANs, decision trees, deep learning \\
\textbf{Statistical Analysis} & Probability theory, information theory, Bayesian methods, statistical modeling \\
\textbf{Programming} & Python, C, MATLAB, Mathematica, statistical computing \\
\textbf{Data Engineering} & Data pipelines, data visualization, feature engineering, model deployment \\
\textbf{Computational Methods} & Monte Carlo simulation, Markov chains, measure-theoretic probability \\
\textbf{Systems} & Linux, Gentoo, OpenBSD, Fedora \\
\textbf{Soft Skills} & Cross-disciplinary collaboration, technical communication, systems thinking
\end{tabular}

  %------------%
  % Experience %
  %------------%

  \section{Experience}

  \headingBf{ST Engineering}{Jan 2024 -- Present}
  \headingIt{Data Scientist (Machine Learning, Statistical Analysis)}{}
  \begin{resume_list}
    \item Implemented backpropagation-based sensitivity analysis using neural networks to identify critical parameters in engineering models, enabling data-driven design optimization.
    \item Applied decision tree learning and ensemble methods to evaluate design parameters and their interactions, guiding experimental design and model simplification.
    \item Developed automated statistical analysis workflows, reducing engineering analysis time from several months to 1 week through ML-driven automation.
    \item Led cross-functional technical discussions to integrate data science solutions into engineering design processes for competitive programs.
  \end{resume_list}

  \headingBf{Advanced Micro Devices ("AMD")}{Jan 2023 -- May 2023}
  \headingIt{Data Scientist and Machine Learning Engineer}{}
  \begin{resume_list}
    \item Built generative adversarial network (GAN)-based surrogate models using PyTorch for complex engineering simulations, achieving 99.5\% runtime reduction.
    \item Developed end-to-end ML pipelines integrating finite element simulations with deep learning workflows, saving over 300 man-hours through automated data processing.
    \item Implemented statistical validation frameworks and experimental design methodologies using advanced computational methods.
  \end{resume_list}

  %-----------%
  % Education %
  %-----------%

  \section{Education}

  \headingBf{Nanyang Technological University, Singapore}{} % Note: Adding year(s) exposes an implied age
  \headingIt{Aerospace Engineering}{}
  \headingIt{Specialisation in Mechanical Engineering, Honours (Highest Distinction), Accelerated Bachelor's}{}

  %------------------------%
  % Technical Competencies %
  %------------------------%

  \section{Technical Competencies}
  
  \begin{resume_list}
    \item \textbf{Deep Learning:} Neural networks, GANs, transfer learning, model optimization, hyperparameter tuning
    \item \textbf{Statistical Modeling:} Bayesian inference, time series analysis, regression analysis, hypothesis testing
    \item \textbf{Data Science:} Feature engineering, model validation, statistical significance testing, reproducible research
    \item \textbf{Computational Statistics:} Monte Carlo methods, stochastic processes, probabilistic modeling, information theory
  \end{resume_list}

\end{document}
\documentclass[letterpaper,10pt]{article}

\usepackage[empty]{fullpage}
\usepackage{titlesec}
\usepackage{enumitem}
\usepackage[hidelinks]{hyperref}
\usepackage{fancyhdr}
\usepackage{fontawesome5}
\usepackage{multicol}
\usepackage{bookmark}
\usepackage{lastpage}

% Sans-Serif
%\usepackage[sfdefault]{FiraSans}
%\usepackage[sfdefault]{roboto}
%\usepackage[sfdefault]{noto-sans}
%\usepackage[default]{sourcesanspro}

% Serif
\usepackage{CormorantGaramond}
\usepackage{charter}

% Colors
% Use with \color{Name}
% Can wrap entire heading with {\color{accent} [...] }
\usepackage{xcolor}
\definecolor{accentTitle}{HTML}{2c446f}
\definecolor{accentText}{HTML}{2c446f}
\definecolor{accentLine}{HTML}{2c446f}

% Misc. Options
\pagestyle{fancy}
\fancyhf{}
\fancyfoot{}
\renewcommand{\headrulewidth}{0pt}
\renewcommand{\footrulewidth}{0pt}
\urlstyle{same}

% Adjust Margins
\addtolength{\oddsidemargin}{-0.7in}
\addtolength{\evensidemargin}{-0.5in}
\addtolength{\textwidth}{1.19in}
\addtolength{\topmargin}{-0.7in}
\addtolength{\textheight}{1.4in}

\setlength{\multicolsep}{-3.0pt}
\setlength{\columnsep}{-1pt}
\setlength{\tabcolsep}{0pt}
\setlength{\footskip}{3.7pt}
\raggedbottom
\raggedright

% ATS Readability
\input{glyphtounicode}
\pdfgentounicode=1

%-----------------%
% Custom Commands %
%-----------------%

% Centered title along with subtitle between HR
% Usage: \documentTitle{Name}{Details}
\newcommand{\documentTitle}[2]{
  \begin{center}
    {\Huge\color{accentTitle} #1}
    \vspace{10pt}
    {\color{accentLine} \hrule}
    \vspace{2pt}
    %{\footnotesize\color{accentTitle} #2}
    \footnotesize{#2}
    \vspace{2pt}
    {\color{accentLine} \hrule}
  \end{center}
}

% Create a footer with correct padding
% Usage: \documentFooter{\thepage of X}
\newcommand{\documentFooter}[1]{
  \setlength{\footskip}{10.25pt}
  \fancyfoot[C]{\footnotesize #1}
}

% Simple wrapper to set up page numbering
% Usage: \numberedPages
% WARNING: Must run pdflatex twice!
\newcommand{\numberedPages}{
  \documentFooter{\thepage/\pageref{LastPage}}
}

% Section heading with horizontal rule
% Usage: \section{Title}
\titleformat{\section}{
  \vspace{-5pt}
  \color{accentText}
  \raggedright\large\bfseries
}{}{0em}{}[\color{accentLine}\titlerule]

% Section heading with separator and content on same line
% Usage: \tinysection{Title}
\newcommand{\tinysection}[1]{
  \phantomsection
  \addcontentsline{toc}{section}{#1}
  {\large{\bfseries\color{accentText}#1} {\color{accentLine} |}}
}

% Indented line with arguments left/right aligned in document
% Usage: \heading{Left}{Right}
\newcommand{\heading}[2]{
  \hspace{10pt}#1\hfill#2\\
}

% Adds \textbf to \heading
\newcommand{\headingBf}[2]{
  \heading{\textbf{#1}}{\textbf{#2}}
}

% Adds \textit to \heading
\newcommand{\headingIt}[2]{
  \heading{\textit{#1}}{\textit{#2}}
}

% Template for itemized lists
% Usage: \begin{resume_list} [items] \end{resume_list}
\newenvironment{resume_list}{
  \vspace{-7pt}
  \begin{itemize}[itemsep=-2px, parsep=1pt, leftmargin=30pt]
}{
  \end{itemize}
  %\vspace{-2pt}
}

% Formats an item to use as an itemized title
% Usage: \itemTitle{Title}
\newcommand{\itemTitle}[1]{
  \item[] \underline{#1}\vspace{4pt}
}

% Bullets used in itemized lists
\renewcommand\labelitemi{--}

%% END_FILE: mteck.sty
%%%%%%%%%%%%%%%


%\numberedPages % NOTE: lastpage requires a second build
%\documentFooter{\thepage of 2} % Does similar without using lastpage
\begin{document}

  %---------%  
  % Heading %
  %---------%

  \documentTitle{Teo Yu Jie}{
    % Web Version
    %\raisebox{-0.05\height} \faPhone\ [redacted - web copy] ~
    %\raisebox{-0.15\height} \faEnvelope\ [redacted - web copy] ~
    %%
    \href{tel:8332 4860}{
      \raisebox{-0.05\height} 8332 4860} ~ | ~
    \href{mailto:yujie@protonmail.com}{
      \raisebox{-0.15\height} yujie@protonmail.com} ~ | ~
    \href{https://www.linkedin.com/in/yujieteo/}{
      \raisebox{-0.15\height} linkedin.com/in/yujieteo/} ~ | ~
    \href{https://github.com/yujieteo?tab=repositories}{
      \raisebox{-0.15\height} https://github.com/yujieteo?tab=repositories}
  }

  %---------%  
  % Summary %
  %---------%

  \tinysection{Summary}
  \begin{resume_list}
    \item Mechanical Engineer with expertise in structural mechanics, finite element analysis, and aerospace systems.
    \item Experienced in structural design, analysis, and optimization for aircraft structural modifications and reliability testing.
    \item Skilled in simulation software and engineering automation, reducing analysis time by 90\% through computational efficiency improvements.
  \end{resume_list}

    %--------%
  % Skills %
  %--------%

\section{Skills}

\begin{tabular}{@{}ll}
\textbf{Simulation Software} &  PyANSYS, ABAQUS, Ansys APDL, Patran/NASTRAN, surrogate modeling \\
\textbf{Analysis Methods} & Finite element analysis, structural mechanics, fatigue analysis, fracture mechanics \\
\textbf{Programming} & Python, MATLAB, Mathematica, C\#, C, FORTRAN \\
\textbf{Technical Areas} & Aeroacoustics, thermomechanics, structural optimization, reliability analysis \\
\textbf{Systems} & Linux, Gentoo, OpenBSD, Fedora \\
\textbf{Soft Skills} & Cross-functional collaboration, technical communication, systems thinking
\end{tabular}

  %------------%
  % Experience %
  %------------%

  \section{Experience}

  \headingBf{ST Engineering}{Jan 2024 -- Present}
  \headingIt{Structural Mechanics Engineer (Stress, Passenger to Freighter Conversions, Engineering Solutions)}{}
  \begin{resume_list}
    \item Conducted sensitivity analysis on 2-DOF aeroacoustics models to identify critical structural parameters, enabling optimized structural design and accelerated design iteration cycles.
    \item Applied statistical analysis methods to evaluate design parameters and structural interactions, guiding engineering decisions for simplified yet robust structural solutions.
    \item Developed C\#/Powershell automation suite for aircraft structural analyses, reducing workflow from several months to 1 week and enabling team to meet critical certification deadlines.
    \item Led technical discussions with cross-functional teams (mechanical, electrical, supply chain) to develop competitive structural design proposals for global aircraft programs.
  \end{resume_list}

  \headingBf{Advanced Micro Devices ("AMD")}{Jan 2023 -- May 2023}
  \headingIt{Mechanical Simulation and Reliability Analysis Intern}{}
  \begin{resume_list}
    \item Integrated finite element simulations using PyANSYS to build surrogate models for board-level reliability analysis, reducing simulation runtime by 99.5\%.
    \item Developed automated simulation workflows and data visualization tools for thermal-mechanical analysis, saving over 300 man-hours within 2 months.
    \item Standardised mechanical testing and characterization workflows using Python and Golang, enabling reproducible experiments and data-driven analysis.
  \end{resume_list}

  %-----------%
  % Education %
  %-----------%

  \section{Education}

  \headingBf{Nanyang Technological University, Singapore}{} % Note: Adding year(s) exposes an implied age
  \headingIt{Aerospace Engineering}{}
  \headingIt{Specialisation in Mechanical Engineering, Honours (Highest Distinction), Accelerated Bachelor's}{}

  %----------------------------%
  % Extracurricular Activities %
  %----------------------------%

  \section{Projects}

  \headingBf{Structural Analysis and Design Projects}{Jan 2024 -- Present}
  \begin{resume_list}
    \item Developed Mathematica applications for advanced mathematical modeling in structural analysis, focusing on probability-based failure analysis and stochastic structural mechanics.
  \end{resume_list}

  \headingBf{Advanced Materials and Fracture Analysis}{Jan 2023 -- Dec 2023}
  \begin{resume_list}
    \item Developed parametric fatigue model for hydrogel materials using advanced constitutive models (Ogden phenomenology) in MATLAB.
    \item Conducted comprehensive fracture simulation and crack characterization studies for inhomogeneous materials, comparing computational results with experimental validation.
    \item Implemented modified phase field methodology for fracture simulation using ABAQUS finite element analysis with custom material modeling in FORTRAN.
  \end{resume_list}

  %------------------------%
  % Technical Competencies %
  %------------------------%

  \section{Technical Competencies}
  
  \begin{resume_list}
    \item \textbf{Structural Analysis:} Static and dynamic analysis, buckling analysis, fatigue life prediction, fracture mechanics
    \item \textbf{Materials Engineering:} Advanced material modeling, composite mechanics, polymer characterization, failure analysis
    \item \textbf{Computational Methods:} Finite element analysis, surrogate modeling, optimization algorithms, Monte Carlo simulation
    \item \textbf{Aerospace Applications:} Aircraft structural modifications, aeroelastic analysis, certification requirements
  \end{resume_list}

\end{document}
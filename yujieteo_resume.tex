
\documentclass[letterpaper,10pt]{article}

\usepackage[empty]{fullpage}
\usepackage{titlesec}
\usepackage{enumitem}
\usepackage[hidelinks]{hyperref}
\usepackage{fancyhdr}
\usepackage{fontawesome5}
\usepackage{multicol}
\usepackage{bookmark}
\usepackage{lastpage}

% Sans-Serif
%\usepackage[sfdefault]{FiraSans}
%\usepackage[sfdefault]{roboto}
%\usepackage[sfdefault]{noto-sans}
%\usepackage[default]{sourcesanspro}

% Serif
\usepackage{CormorantGaramond}
\usepackage{charter}

% Colors
% Use with \color{Name}
% Can wrap entire heading with {\color{accent} [...] }
\usepackage{xcolor}
\definecolor{accentTitle}{HTML}{2c446f}
\definecolor{accentText}{HTML}{2c446f}
\definecolor{accentLine}{HTML}{2c446f}

% Misc. Options
\pagestyle{fancy}
\fancyhf{}
\fancyfoot{}
\renewcommand{\headrulewidth}{0pt}
\renewcommand{\footrulewidth}{0pt}
\urlstyle{same}

% Adjust Margins
\addtolength{\oddsidemargin}{-0.7in}
\addtolength{\evensidemargin}{-0.5in}
\addtolength{\textwidth}{1.19in}
\addtolength{\topmargin}{-0.7in}
\addtolength{\textheight}{1.4in}

\setlength{\multicolsep}{-3.0pt}
\setlength{\columnsep}{-1pt}
\setlength{\tabcolsep}{0pt}
\setlength{\footskip}{3.7pt}
\raggedbottom
\raggedright

% ATS Readability
\input{glyphtounicode}
\pdfgentounicode=1

%-----------------%
% Custom Commands %
%-----------------%

% Centered title along with subtitle between HR
% Usage: \documentTitle{Name}{Details}
\newcommand{\documentTitle}[2]{
  \begin{center}
    {\Huge\color{accentTitle} #1}
    \vspace{10pt}
    {\color{accentLine} \hrule}
    \vspace{2pt}
    %{\footnotesize\color{accentTitle} #2}
    \footnotesize{#2}
    \vspace{2pt}
    {\color{accentLine} \hrule}
  \end{center}
}

% Create a footer with correct padding
% Usage: \documentFooter{\thepage of X}
\newcommand{\documentFooter}[1]{
  \setlength{\footskip}{10.25pt}
  \fancyfoot[C]{\footnotesize #1}
}

% Simple wrapper to set up page numbering
% Usage: \numberedPages
% WARNING: Must run pdflatex twice!
\newcommand{\numberedPages}{
  \documentFooter{\thepage/\pageref{LastPage}}
}

% Section heading with horizontal rule
% Usage: \section{Title}
\titleformat{\section}{
  \vspace{-5pt}
  \color{accentText}
  \raggedright\large\bfseries
}{}{0em}{}[\color{accentLine}\titlerule]

% Section heading with separator and content on same line
% Usage: \tinysection{Title}
\newcommand{\tinysection}[1]{
  \phantomsection
  \addcontentsline{toc}{section}{#1}
  {\large{\bfseries\color{accentText}#1} {\color{accentLine} |}}
}

% Indented line with arguments left/right aligned in document
% Usage: \heading{Left}{Right}
\newcommand{\heading}[2]{
  \hspace{10pt}#1\hfill#2\\
}

% Adds \textbf to \heading
\newcommand{\headingBf}[2]{
  \heading{\textbf{#1}}{\textbf{#2}}
}

% Adds \textit to \heading
\newcommand{\headingIt}[2]{
  \heading{\textit{#1}}{\textit{#2}}
}

% Template for itemized lists
% Usage: \begin{resume_list} [items] \end{resume_list}
\newenvironment{resume_list}{
  \vspace{-7pt}
  \begin{itemize}[itemsep=-2px, parsep=1pt, leftmargin=30pt]
}{
  \end{itemize}
  %\vspace{-2pt}
}

% Formats an item to use as an itemized title
% Usage: \itemTitle{Title}
\newcommand{\itemTitle}[1]{
  \item[] \underline{#1}\vspace{4pt}
}

% Bullets used in itemized lists
\renewcommand\labelitemi{--}

%% END_FILE: mteck.sty
%%%%%%%%%%%%%%%%%%%%%%


%\numberedPages % NOTE: lastpage requires a second build
%\documentFooter{\thepage of 2} % Does similar without using lastpage
\begin{document}

  %---------%
  % Heading %
  %---------%

  \documentTitle{Teo Yu Jie}{
    % Web Version
    %\raisebox{-0.05\height} \faPhone\ [redacted - web copy] ~
    %\raisebox{-0.15\height} \faEnvelope\ [redacted - web copy] ~
    %%
    \href{tel:8332 4860}{
      \raisebox{-0.05\height} 8332 4860} ~ | ~
    \href{mailto:yujie@protonmail.com}{
      \raisebox{-0.15\height} yujie@protonmail.com} ~ | ~
    \href{https://www.linkedin.com/in/yujieteo/}{
      \raisebox{-0.15\height} linkedin.com/in/yujieteo/} ~ | ~
    \href{https://github.com/yujieteo?tab=repositories}{
      \raisebox{-0.15\height} https://github.com/yujieteo?tab=repositories}
  }

  %---------%
  % Summary %
  %---------%

  \tinysection{Summary}
  \begin{resume_list}
    \item Structural Mechanics Engineer applying machine learning to mechanical and aerospace systems.
    \item Experienced in physics-based modeling and ML-driven optimization for aeroacoustics, thermomechanics, and structural analysis.
    \item Skilled in automating simulation pipelines and cross-disciplinary collaboration, enabling 90 \% faster analyses and more scalable design workflows.
  \end{resume_list}

    %--------%
  % Skills %
  %--------%

\section{Skills}

\begin{tabular}{@{}ll}
\textbf{Machine Learning \,\,} &  PyTorch, tinygrad, NumPy, SciPy, pandas, GANs, decision trees \\
\textbf{Simulation} & PyANSYS, ABAQUS, Ansys APDL, Patran/NASTRAN, surrogate modeling \\
\textbf{Programming} & C\#, C, Python, MATLAB, Mathematica \\
\textbf{Mathematics} & Linear algebra and representation theory, probability theory, information theory \\
\textbf{Systems} & Linux, Gentoo, OpenBSD, Fedora \\
\textbf{Soft Skills} & Cross-disciplinary collaboration, communication, systems thinking
\end{tabular}

  %------------%
  % Experience %
  %------------%

  \section{Experience}

  \headingBf{ST Engineering}{Jan 2024 -- Present}
  \headingIt{Structural Mechanics Engineer (Stress, Passenger to Freighter Conversions, Engineering Solutions)}{}
  \begin{resume_list}
    \item Implemented backpropagation-based sensitivity analysis on a 2-DOF aeroacoustics model to identify critical structural/acoustic parameters, enabling targeted model simplification and accelerating design iteration.
    \item Applied decision tree learning to evaluate parameter weightage for aeroacoustic design and interactions, guiding model simplification and experimental design.
    \item Developed C\#/Powershell automation suite integrating with Microsoft Word/Excel/PowerPoint for rapid delivery of aircraft structural analyses (local/global static instability of frames), reducing workload from several months of analyses backlog to 1 week enabling team to meet certification deadlines.
    \item Led cross-functional technical discussions (mechanical, electrical, supply chain) to shape structural design proposals for globally competitive programs.
  \end{resume_list}

  \headingBf{Advanced Micro Devices ("AMD")}{Jan 2023 -- May 2023}
  \headingIt{Machine Learning and Thermomechanics Simulation Intern}{}
  \begin{resume_list}
    \item Integrated finite element simulations with Python using PyANSYS to build a generative adversarial network (GAN)-based surrogate model for board-level reliability analysis, reducing simulation runtime by 99.5\%.
    \item Developed open source automated simulation, ML pipelines, and architectural data workflows and internal visualisation tools for board level temperature cycling, saving over 300 man-hours within 2 months and cutting computation time by 99.5\%.
    \item Standardised dynamic mechanical analysis + digital image correlation workflows in Python and Golang, enabling reproducible experiments and ML-ready data pipelines.
  \end{resume_list}

  %-----------%
  % Education %
  %-----------%

  \section{Education}

  \headingBf{Nanyang Technological University, Singapore}{} % Note: Adding year(s) exposes an implied age
  \headingIt{Aerospace Engineering}{}
  \headingIt{Specialisation in Mechanical Engineering, Honours (Highest Distinction), Accelerated Bachelor's}{}

  %----------------------------%
  % Extracurricular Activities %
  %----------------------------%

  \section{Projects}

  \headingBf{Mathematica Hobby Projects}{Jan 2024 -- Present}
  \begin{resume_list}
    \item Developed Mathematica notebooks exploring Monte Carlo, Markov chains, and measure-theoretic probability.
  \end{resume_list}

  \headingBf{Crack chracterisation for hydrogel fracture simulation}{Jan 2023 -- Dec 2023}
  \begin{resume_list}
    \item Developed parametric fatigue model using Ogden hydrogel phenomenology in MATLAB.
    \item Simulated fracture and characterised crack morphology of inhomogenous hydrogel against experimental results using MATLAB generated material input files (ABAQUS) and meshing using ABAQUS and FORTRAN. Simulation methodology is using an experimental modified phase field methodology based on nodal temperature. 
  \end{resume_list}


\end{document}